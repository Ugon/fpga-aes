\section{Przyjęta metodyka pracy}
\label{sec:przyjeta-metodyka-pracy}
Proces tworzenia projektu prowadzony był zgodnie z iteracyjnym modelem wytwarzania oprogramowania. Prace zostały podzielone na szereg iteracji, z których każda miała na celu rozbudowę poprzedniej wersji o nowe funkcjonalności. Produkt po każdym przyroście miał poprawnie realizować dodany fragment funkcjonalności oraz być przetestowany.

Ze względu na trudności w oszacowaniu, jak długo mogą zająć kolejne iteracje, wyszczególnione zostały dwa duże etapy agregujące po kilka iteracji. Każdy z tych etapów ma nieprzekraczalny termin. Drugi etap ma na celu dostarczenie gotowego produktu oraz po jego zakończeniu przewidziany jest długi okres czasu na ewentualne poprawki, oraz finalizację projektu.

Zdecydowano się na wykorzystanie przyrostowego modelu wytwarzania oprogramowania, ponieważ wstępna analiza problemu pokazała, że da się go w naturalny sposób podzielić na kilka iteracji o porównywalnej ilości prac do wykonania. Ponadto zespół projektowy miał już okazję pracować zgodnie z takim modelem, co pokazało że podział pracy na etapy ułatwia pracę nad projektem.

\subsection{Pierwszy etap projektu}
Pierwszy etap ma na celu implementację komunikacji między komputerem a płytką FPGA wraz z obsługą błędów. Po jego zakończeniu układ FPGA ma być zdolny odbierać strumień bloków oraz odsyłać je z powrotem bez żadnych modyfikacji. Błędy transmisji powinny być wykrywane i obsługiwane. Implementacja ma być przystosowana do łatwego zintegrowania z modułem szyfrującym AES oraz sposobem wiązania bloków CBC. Etap pierwszy kończy się 30.06.2016r (koniec VI semestru).

\subsubsection{Planowane iteracje pierwszego etapu projektu}
\begin{itemize}
\item Zebranie informacji na temat układu FTDI i jego sterownikach, sprawdzenie czy jego wykorzystanie do realizacji komunikacji między komputerem a płytką będzie możliwe.
\item Założenie projektu Quartus, skonfigurowanie pinów, wyprowadzenie sygnałów z układu FTDI do układu FPGA (przy użyciu programu Qsys). Zaimplementowanie modułu odbierającego znaki i odsyłającego je z powrotem.
\item Zaimplementowanie modułów odpowiedzialnych za serializację i deserializację pomiędzy sygnałem UART a blokiem danych o wielkości 128b.
\item Zaimplementowanie pierwszej wersji maszyny stanów odpowiedzialnej za sterowanie procesem odbierania i przesyłania bloków.
\item Dodanie kontroli błędów przy użyciu sumy kontrolnej CRC16.
\end{itemize}

\subsection{Drugi etap projektu}
Drugi etap ma na celu implementację szyfrowania i deszyfrowania AES, sposobu wiązania bloków CBC oraz dostarczenie gotowego produktu. Kończy się 30.09.2016r (koniec wakacji).
\subsubsection{Planowane iteracje drugiego etapu projektu}
\begin{itemize}
\item Implementacja modułu szyfrującego blok AES.
\item Integracja modułu szyfrującego AES oraz implementacja i integracja sposobu łączenia bloków CBC.
\item Implementacja modułu deszyfrującego blok AES.
\item Integracja modułu deszyfrującego AES z uwzględnieniem sposobu łączenia bloków CBC.
\item Przygotowanie karty SD zawierającej skrypt startowy programujący układ FPGA.
\end{itemize}

\subsection{Narzędzia użyte do zarządzania projektem}
Do przechowywonia i wersjoniwania kodu źródłowego użyto systemu kontroli wersji Git oraz publicznego repozytorium kodu na stronie \textit{www.github.com}. Pozwoliło to na wygodne zarządzanie kodem źródłowym, wgląd w historię oraz inspekcję po wprowadzeniu zmian do repozytorium. Takie rozwiązanie daje również możliwość łatwego wglądu w kod dla promotora oraz opiekuna projektu.

Ze względu na fakt, że zespół projektowy był jednoosobowy, nie było konieczności używania innych narzędzi do zarządzania i wspomagania procesu wytwarzania oprogramowania. Wszelkie notatki, schematy oraz plany były tworzone odręcznie lub przy pomocy prostych edytorów tekstowych oraz narzędzi graficznych. Takie informacje były przechowywane lokalnie, oraz w razie konieczności były udostępnianie w repozytorium kodu lub wysyłane pocztą elektroniczną.


\newpage