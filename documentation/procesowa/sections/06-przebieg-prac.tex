\section{Przebieg prac}
\label{sec:przebieg-prac}
Prace przebiegały zgodnie z zaplanowanym harmonogramem. Pierwszy etap implementacji udało się zakończyć przed planowanym terminem, w wyniku czego pozostało więcej czasu na realizację etapu drugiego.

\subsection{Szczegółowy harmonogram prac}
\begin{description}
\item[do 31.03.2016] Wybranie tematu projektu inżynierskiego, ustalenie wstępnej wizji produktu i wymagań. Wybór i zamówienie płytki z układem FPGA.

\item[01.04.2016 - 17.04.2016] Potwierdzenie możliwości wykorzystania konwertera USB-UART firmy FTDI znajdującego się na płytce. Konfiguracja pinów płytki. Znalezienie sposobu na podłączenie sygnałów RX i TX z układu FTDI do FPGA - stworzenie projektu QSys oraz wygenerowanie preloadera, który przy starcie odpowiednio konfiguruje piny UART. Utworzenie karty SD zawierającej preloader.

\item[18.04.2016 - 24.04.2016] Zbieranie szczegółowych informacji na temat protokołu UART, szyfrowania AES, sposobu wiązania bloków CBC. Planowanie realizacji projektu:
	\begin{itemize}[noitemsep,nolistsep]
	\item Plan implementacji modułów odbierających i wysyłających bajty
	\item Plan implementacji modułów serializujących i deserializujących bloki danych
	\item Wstępny plan maszyny stanów sterującej procesem odbierania i wysyłania bloków danych uwzględniający wiązanie bloków CBC
	\item Plan sposobu przyszłej integracji modułów szyfrujących i deszyfrujących AES
	\end{itemize}

\item[25.04.2016 - 28.04.2016] Implementacja modułu odbierającego i odsyłającego pojedyncze bajty.

\item[29.04.2016 - 30.04.2016] Implementacja modułów serializacji i deserializacji 128 bitowych bloków danych.

\item[01.05.2016 - 03.05.2016] Implementacja maszyny stanów sterującej procesem odbierania i wysyłania bloków danych. Zaobserwowanie przekłamań podczas transmisji danych.

\item[04.05.2016 - 07.05.2016] Opracowanie protokołu transmisji uwzględniającego możliwość wystąpienia błędów, wykorzystującego obliczanie sumy kontrolnej CRC do ich wykrywania. Protokół ten wykorzystuje przesyłanie dodatkowych bajtów ACK i NACK pomiędzy blokami oraz obsługuje retransmisje bloków w przypadku wykrycia błędów. Implementacja opracowanego rozwiązania oraz integracja z istniejącą maszyną stanów.

\item[08.05.2016 - 12.05.2016] Implementacja bloku szyfrującego AES.

\item[13.05.2016 - 14.05.2016] Implementacja sposobu wiązania bloków CBC oraz integracja bloku szyfrującego AES.

\item[01.09.2016 - 14.09.2016] Implementacja bloku deszyfrującego AES, oraz jego integracja (z uwzględnieniem sposobu wiązania bloków CBC).

\item[15.09.2016 - 30.09.2016] Stworzenie karty SD zawierającej skrypt programujący układ FPGA przy starcie.
\end{description}


\subsection{Napotkane problemy}
Napotkanym wyzwaniem, którego się nie spodziewałem okazała się odpowiednia konfiguracja płytki FPGA. Problemem był fakt, że wybrany przeze mnie model ma zintegrowany procesor ARM, do którego podłączony jest układ FTDI, do którego sygnałów potrzebowałem mieć dostęp z części FPGA układu. Instrukcje obsługi dołączone przez producenta płytki nie dostarczały wystarczających informacji jak to zrealizować, a źródła znalezione w internecie (wpisy użytkowników na forum) były lakoniczne i nieraz sprzeczne. Spowodowało to, że musiałem poświęcić dużo czasu, aby metodą prób i błędów dojść do rozwiązania.
\newline
Kolejnym problemem, również związanym z odpowiednią konfiguracją płytki, było utworzenie karty SD zawierającej skrypt programujący układ FPGA przy starcie. Tym razem trudnością okazał się fakt, że potrzebowałem wykorzystać układ UART, przez który w czasie uruchamiania wysyłane są informacje traktujące o wykonywanych krokach i raportowane ewentualne błędy. Spowodowało to, że w czasie prac nad skryptem startowym nie miałem możliwości podglądu, co się dzieje. Podobnie jak w przypadku pierwszego problemu, informacje które udało mi się znaleźć nie były wystarczające do szybkiego rozwiązania problemu i ponownie musiałem poświęcić dużo czasu na próby i eksperymenty.
\newline
Trzecim z dużych problemów, które napotkałem było niedeterministyczne zachowanie głównej maszyny stanów -- wchodzenie w niezdefiniowany stan i zawieszanie. Z powodu braku doświadczenia trudno mi znaleźć przyczynę problemu, lub nawet sporządzić listę potencjalnych punktów systemu, które mogą być odpowiedzialne za błąd. Powodem problemu okazał się brak synchronizacji sygnału UART RX pochodzącego z zewnętrznego układu (konwertera USB-UART firmy FTDI). Powodowało to, że podczas rosnących lub malejących zboczy zegara, które wyzwalały m.in. zmiany stanów maszyny, sygnał UART RX mógł nie mieć dobrze zdefiniowanej wartości (np. również być w trackie rosnącego lub malejącego zbocza), co w efekcie prowadziło do niedeterministycznego zachowania układu. Rozwiązaniem problemu było dodanie synchronizacji sygnału pochodzącego z urządzenia peryferyjnego poprzez dodanie na jego drodze dwóch przerzutników typu D wyzwalanych zboczami głównego zegara \cite{altera-metastability, 2ff-synchronization}, oraz zdefiniowanie ograniczeń czasowych układu (ang. \textit{timing constraints}). Głównym powodem trudności w rozwiązaniu problemu było jego znalezienie, ze względu na niedeterministyczną naturę.


\newpage