\section{Cel projektu}
\label{sec:cel-projektu}

\subsection{Motywacja projektu}
Od wielu wieków istnieje potrzeba ochrony tajnych informacji przed przechwyceniem przez osoby niepowołane. Znane są przykłady takich czynności pochodzące już z czasów starożytnych. Pierwszym powszechnie znanym przykładem jest szyfr Cezara, który był używany przez rzymskiego wodza do ochrony poufnej korespondencji. Innym dobrze znanym przykładem jest szyfr Enigmy, który odegrał ważną rolę podczas II wojny światowej.
\newline
Początkowo szyfrowanie było używane w nielicznych przypadkach, obecnie towarzyszy ludziom każdego dnia. Największą domeną, w której stosowana jest kryptografia jest niewątpliwie ochrona danych cyfrowych. Wykorzystywana jest m.in. szyfrowania komunikacji i danych zapisanych na różnych nośnikach, do uwierzytelniania, jest podstawą walut elektronicznych itp.
\newline
Ze względu na postęp technologiczny, stare metody szyfrowania nie są już bezpieczne. Przykładem może być szyfr DES, który został zatwierdzony jako standard w 1976 roku. Od tamtej pory moc obliczeniowa komputerów na tyle się zwiększyła, że obecnie DES można złamać przy pomocy ataków \textit{brute force} i jest on obecnie uważany za algorytm niezapewniający odpowiedniego bezpieczeństwa.
\newline
W celu zapewnienia bezpieczeństwa stosowane współcześnie szyfry wymagają wykonania wielu operacji aby dane zaszyfrować oraz odszyfrować. W połączeniu z rosnąca popularnością kryptografii, jest to powodem dla przystosowywania istniejących procesorów do szybkiego wykonywania instrukcji szyfrujących, lub nawet tworzenia osobnych szyfrujących układów sprzętowych.

\subsection{Cel projektu i wizja produktu}
Celem mojego projektu inżynierskiego jest zaimplementowanie sprzętowego układu szyfrującego algorytmem AES na platformie FPGA, oraz praktyczne zaprezentowanie jego działania. Końcowy produkt powinien umożliwić użytkownikowi szyfrowanie i deszyfrowanie plików. Sposób szyfrowanie powinien być zgodny z właściwymi standardami oraz produkt powinien być kompatybilny z obecnie istniejącymi rozwiązaniami.
\newline
Produkt końcowy będzie program konsolowy przystosowany do uruchomienia na systemach operacyjnych Ubuntu, oraz karta SD zawierająca skrypt programujący układ FPGA. Do korzystania z produktu konieczne będzie połączenie komputera z płytką kablem USB. Produkt będzie współpracował z płytką Terasic DE1-SOC.
\newline
Płytka Terasic DE1-SOC będzie komunikować się z komputerem przez kabel USB. Protokołem komunikacji będzie UART tunelowany przez kabel USB. Na płytce zainstalowany jest konwerter umożliwiający konwersję USB-UART firmy FTDI. W systemach operacyjnych Ubuntu zintegrowany jest sterownik współpracujący z tym układem, który nie wymaga dodatkowej instalacji.
\newline
W celu zaszyfrowania pliku na komputerze będzie uruchamiany program będący częścią produktu, który będzie odczytywał z dysku kolejne bloki danych w postaci jawnej, wysyłał je do układu FPGA oraz odbierał bloki danych zaszyfrowanych i zapisywał je do innego pliku. Do układu FPGA będą doprowadzone sygnały RX i TX protokołu UART wyprowadzone z układu FTDI. Układ będzie odbierał bloki danych, szyfrował je i odsyłał ich zaszyfrowaną postać z powrotem. Deszyfrowanie będzie przebiegać analogicznie. Ze względu na możliwość wystąpienia błędów podczas komunikacji UART, zaimplementowana zostanie obsługa błędów oparta o sumy kontrolne CRC.
\newline
Programowanie układu FPGA będzie zrealizowanie w taki sposób, aby w celu uruchomienia produktu użytkownik nie musiał posiadać wiedzy technicznej. Włączenie płytki w taki sposób aby układ FPGA został odpowiednio zaprogramowany będzie wymagało przestawiania kilku przełączników oraz włożenia dostarczonej karty SD, która będzie zawierać odpowiedni skrypt startowy realizujący programowanie układu.

\newpage