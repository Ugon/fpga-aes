\section{Analiza ryzyka}
\label{sec:analiza-ryzyka}
Proces tworzenia produktu wiąże się z następującymi zagrożeniami:
\begin{itemize}
\item Zbyt mała liczba jednostek logicznych w układzie FPGA. Moduły szyfrowania i deszyfrowania AES mogą się okazać zbyt złożone, aby mogły się zmieścić w wybranym układzie FPGA. W takim przypadku konieczne będzie zmniejszenie długości klucza, lub dostarczenie dwóch osobnych produktów - szyfrującego i deszyfrującego.
\item Brak możliwości użycia układu UART znajdującego się na płytce, które mogą prowadzić do konieczności wykorzystania osobnego konwertera USB-UART. Prowadziłoby to do utrudnienia dla użytkownika końcowego związane z brakiem możliwości połączenia płytki z komputerem przy pomocy jednego kabla USB.
\item Brak możliwości uruchomienia płytki w taki sposób, aby układ FPGA został skonfigurowany przy starcie. Prowadziłoby to do utrudnienia dla użytkownika końcowego związanego z koniecznością ręcznego programowania płytki.
\item Problemy związane z niezerowym czasem propagacji sygnału oraz innymi właściwościami fizycznymi układu FPGA, powodującymi niedeterministyczne oraz trudne do wykrycia błędy.
\item Czynniki losowe opóźniające pracę lub uniemożliwiające ukończenie projektu na czas, np. choroba.
\item Uszkodzenie sprzętu ze względu na wadę produkcyjną lub niekompetencję zespołu projektowego.
\end{itemize}



\newpage