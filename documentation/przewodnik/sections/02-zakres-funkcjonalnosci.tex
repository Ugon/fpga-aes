\section{Zakres funkcjonalności}
\label{sec:zakres-funkcjonalnosci}

Produkt jest przeznaczony dla użytkowników komputerów chcących zabezpieczyć swoje pliki przed odczytaniem przez niepowołane osoby. System dostarczy możliwości szyfrowania danych algorytmem AES przy użyciu klucza o długości 256b podanego przez użytkownika. Program będzie łatwy w użyciu.

Jedynym \textbf{aktorem} w kontekście systemu jest użytkownik końcowy chcący zaszyfrować swój plik. Produkt nie wchodzi w interakcję z żadnym zewnętrznym systemem.

\subsection{Wymagania funkcjonalne}
\begin{enumerate}
\item System umożliwia szyfrowanie i deszyfrowanie plików
\item Szyfrowanie bloków zgodne ze standardem NIST FIPS-197 \cite{aes-standard}
\item Szyfrowanie kluczem o długości 256b
\item Tryb wiązania bloków zaszyfrowanych CBC zgodny ze standardem NIST SP 800-38A \cite{cbc-standard}
\item Sposób szyfrowania powinien być kompatybilny z programem \textit{openssl aes-256-cbc} 
\item Transmisja danych między komputerem a układem FPGA powinna być realizowana w sposób wolny od błędów
\end{enumerate}

\subsection{Wymagania niefunkcjonalne}
\begin{enumerate}
\item System jest łatwy w obsłudze
\item Program współpracuje z systemem operacyjnym Ubuntu
\item Sposób uruchomienia programu analogiczny do programu \textit{openssl}
\item Dostarczenie dokumentacji technicznej, procesowej i użytkownika

\end{enumerate}



\newpage