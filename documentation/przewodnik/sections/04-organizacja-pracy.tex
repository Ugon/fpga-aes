\section{Organizacja pracy}
\label{sec:organizacja-pracy}

% \emph{Struktura zespołu (role poszczególnych osób), krótki opis i
  % uzasadnienie przyjętej metodyki i/lub kolejności prac, planowane i
  % zrealizowane etapy prac ze wskazaniem udziału poszczególnych
  % członków zespołu, wykorzystane praktyki i narzędzia w zarządzaniu
  % projektem.}
Przed przystąpieniem do realizacji projektu przeprowadzono fazę planowania, w której został określony plan organizacji pracy.

\subsection{Struktura zespołu -- role oraz odpowiedzialności}
Zespół projektowy składa się z jednej osoby -- autora pracy inżynierskiej. Opierając się na wstępnej analizie problemu można wyróżnić następujące role i ich odpowiedzialności:
\begin{description}
\item[Kierownik projektu] -- stworzenie harmonogramu prac oraz kontrola jego realizacji
\item[Lider techniczny] -- wybór technologii użytych przy realizacji systemu
\item[Analityk] -- kontakt z klientem, uzgodnienie i sformalizowanie wymagań funkcjonalnych i niefunkcjonalnych produktu
\item[Projektant] -- sporządzenie projektu technicznego systemu: dobór architektury systemu, projekt komponentów oraz interfejsów pomiędzy nimi
\item[Programista] -- implementacja systemu
\item[Tester] -- przetestowanie systemu
\item[Twórca dokumentacji] -- stworzenie dokumentacji
\end{description}

\subsection{Metodyka pracy}
Proces tworzenia projektu prowadzony był zgodnie z iteracyjnym modelem wytwarzania oprogramowania. Prace zostały podzielone na szereg iteracji, z których każda miała na celu rozbudowę poprzedniej wersji o nowe funkcjonalności. Efekt każdego przyrostu miał być funkcjonalny oraz przetestowany.


Zdecydowano się na wykorzystanie przyrostowego modelu wytwarzania oprogramowania, ponieważ wstępna analiza problemu pokazała, że da się go w naturalny sposób podzielić na kilka iteracji o porównywalnej ilości prac do wykonania. Ponadto zespół projektowy miał już okazję pracować zgodnie z takim modelem, co pokazało że podział pracy na etapy ułatwia pracę nad projektem.

\subsection{Harmonogram prac}
Przed przystąpieniem do prac opracowano harmonogram. Ze względu na trudności w oszacowaniu, jak długo mogą zająć kolejne iteracje, wyszczególnione zostały dwa duże etapy agregujące po kilka przyrostów. Każdy z tych etapów ma nieprzekraczalny termin. Drugi etap ma na celu dostarczenie gotowego produktu oraz po jego zakończeniu przewidziany jest długi okres czasu na ewentualne poprawki, oraz finalizację projektu.

Prace przebiegały zgodnie z zaplanowanym harmonogramem. Pierwszy etap implementacji udało się zakończyć przed planowanym terminem, w wyniku czego pozostało więcej czasu na realizację etapu drugiego. Szczegółowy terminarz prac można znaleźć w dokumentacji procesowej.

\subsubsection{Pierwszy etap projektu}
Pierwszy etap ma na celu implementację komunikacji między komputerem a płytką FPGA wraz z obsługą błędów. Po jego zakończeniu układ FPGA ma być zdolny odbierać strumień bloków oraz odsyłać je z powrotem bez żadnych modyfikacji. Błędy transmisji powinny być wykrywane i obsługiwane. Implementacja ma być przystosowana do łatwego zintegrowania z modułem szyfrującym AES oraz sposobem wiązania bloków CBC. Etap pierwszy kończy się 30.06.2016r (koniec VI semestru).

\paragraph{Iteracje pierwszego etapu projektu}
\begin{enumerate}
\item Zebranie informacji na temat konwertera USB-UART firmy FTDI i jego sterownikach, sprawdzenie czy jego wykorzystanie do realizacji komunikacji między komputerem a płytką będzie możliwe.
\item Założenie projektu Quartus, skonfigurowanie pinów, wyprowadzenie sygnałów z układu FTDI do układu FPGA (przy użyciu narzędzia Qsys). Zaimplementowanie modułu odbierającego znaki i odsyłającego je z powrotem.
\item Zaimplementowanie modułów odpowiedzialnych za serializację i deserializację pomiędzy sygnałem UART a blokiem danych o wielkości 128b.
\item Zaimplementowanie pierwszej wersji maszyny stanów odpowiedzialnej za sterowanie procesem odbierania i przesyłania bloków.
\item Dodanie kontroli błędów przy użyciu sumy kontrolnej CRC16.
\end{enumerate}

\subsubsection{Drugi etap projektu}
Drugi etap ma na celu implementację szyfrowania i deszyfrowania AES, sposobu wiązania bloków CBC oraz dostarczenie gotowego produktu.Kończy się 30.09.2016r (koniec wakacji).
\paragraph{Iteracje drugiego etapu projektu}
\begin{enumerate}
\item Implementacja modułu szyfrującego blok AES.
\item Integracja modułu szyfrującego AES oraz implementacja i integracja sposobu łączenia bloków CBC.
\item Implementacja modułu deszyfrującego blok AES.
\item Integracja modułu deszyfrującego AES z uwzględnieniem sposobu łączenia bloków CBC.
\item Przygotowanie karty SD zawierającej skrypt startowy programujący układ FPGA.
\end{enumerate}

\subsection{Narzędia użyte do zarządzania projektem}
Do przechowywonia i wersjoniwania kodu źródłowego użyto systemu kontroli wersji Git oraz publicznego repozytorium kodu na stronie \textit{www.github.com}. Pozwoliło to na wygodne zarządzanie kodem źródłowym, wgląd w historię oraz inspekcję po wprowadzeniu zmian do repozytorium. Takie rozwiązanie daje również możliwość łatwego wglądu w kod dla promotora oraz opiekuna projektu.

Ze względu na fakt, że zespół projektowy był jednoosobowy, nie było konieczności używania innych narzędzi do zarządzania i wspomagania procesu wytwarzania oprogramowania. Wszelkie notatki, schematy oraz plany były tworzone odręcznie lub przy pomocy prostych edytorów tekstowych oraz narzędzi graficznych. Takie informacje były przechowywane lokalnie, oraz w razie konieczności były udostępnianie w repozytorium kodu lub wysyłane pocztą elektroniczną.