\subsubsection{Moduł deserializujący bajty UART do bloków AES}
Interfejs modułu deserializującego bajty UART do bloków AES \textit{block\_deserializer}:
\begin{interface}{RX\_FINISHED\_LISTENING}
	\item[\insignal{CLK\_16}] główny zegar układu.

	\item[\insignal{RX\_BYTE[7:0]}] sygnał pochodzący z komponentu \textit{uart\_rx}.
	\item[\outsignal{RX\_START\_LISTENING}] sygnał wychodzący do komponentu \textit{uart\_rx}.
	\item[\insignal{RX\_FINISHED\_LISTENING}] sygnał pochodzący z komponentu \textit{uart\_rx}.

	\item[\insignal{START\_LISTENING}] sygnał wprowadzający komponent w stan oczekiwania na blok.
	\item[\outsignal{BLOCK[127:0]}] odebrany blok AES. Stabilność sygnału jest gwarantowana, gdy \outsignal{FINISHED\_LISTENING} jest w stanie wysokim.
	\item[\outsignal{FINISHED\_LISTENING}] sygnał informujący o zakończeniu odbierania bloku AES oraz gotowość na otrzymanie kolejnego sygnału \insignal{START\_LISTENING} podczas następnego zbocza rosnącego zegara \insignal{CLK\_16}.
	\item[\outsignal{CORRECT}] sygnał określający, czy transmisja bloku AES przebiegła bez błędów. Stabilność sygnału jest gwarantowana, gdy \outsignal{FINISHED\_LISTENING} jest w stanie wysokim.
\end{interface}

Po otrzymaniu sygnału \insignal{START\_LISTENING} moduł rozpoczyna nasłuchiwanie na transmisję. Odbieranie bajtów bloku danych wykonywane jest przez przez moduł \textit{uart\_rx}. Kolejność ułożenia bajtów w bloku \outsignal{BLOCK[127:0]} jest zgodna ze standardem AES. W trakcie odbierania obliczana jest suma kontrolna \textit{CRC16}. Po 128 bajtach bloku danych przesyłane są 2 bajty zawierające oczekiwaną, obliczoną po stronie klienta sumę kontrolną. Jeśli jest ona zgodna z tą obliczaną na bieżąco przez moduł, sygnał \outsignal{CORRECT} przyjmuje wartość {'1'}, w przeciwnym wypadku wartość {'0'}. Po odebraniu 130 bajtów moduł zwraca blok \outsignal{BLOCK[127:0]} wraz z informacją o jego poprawności \outsignal{CORRECT} oraz sygnalizuje gotowość na rozpoczęcie nasłuchiwania na kolejne bloki \outsignal{FINISHED\_LISTENING}.

\begin{figure}[!h]
	\centering
	\begin{tikztimingtable}[timing/wscale=0.95]
	\insignal{CLK\_16}          & 3{cc}  16{cc}           16{cc}         \\
	\outsignal{RX\_START\_LIST} & LH     15LH             15LH           L\\
	\insignal{RX\_BYTE[7:0]}    & L      15LD{BLOCK[0]}   15LD{BLOCK[1]} 2L\\
	\insignal{RX\_FIN\_LIST}    & L      15LH             15LH           2L\\
	\insignal{START\_LIST}      & LH     16L              16L            L\\
	\extracode
	\tablerules
	\end{tikztimingtable}
\caption{\textit{block\_deserializer} -- rozpoczęcie odbierania}
\end{figure}

\begin{figure}[!h]
	\centering
	\begin{tikztimingtable}[timing/wscale=0.95]
	\insignal{CLK\_16}          & 3{cc}          16{cc}       16{cc}             \\
	\outsignal{RX\_START\_LIST} & 2LH            15LH         16L                \\
	\insignal{RX\_BYTE[7:0]}    & LD{BLOCK[127]} 15LD{CRC[0]} 15LD{CRC[1]}       L\\
	\insignal{RX\_FIN\_LIST}    & LH             15LH         15LH               L\\
	\outsignal{BLOCK[127:0]}    & 3L             15L          15LD{BLOCK[127:0]} L\\
	\outsignal{FIN\_LIST}       & 3L             15L          15LH               L\\
	\outsignal{CORRECT}         & 3L             15L          15LH               L\\
	\extracode
	\tablerules
	\end{tikztimingtable}
\caption{\textit{block\_deserializer} -- zakończenie odbierania}
\end{figure}

\begin{figure}[!h]
	\centering
	\begin{tikztimingtable}
	\helpsignal{CLK\_UART}      & 1.5l  130{0.25c0.25c}    l \\
	\outsignal{RX\_START\_LIST} & l     130{0.5lg}      1.5l \\
	\insignal{RX\_BYTE[7:0]}    & 1.5l  130{0.5lg}         l \\
	\insignal{RX\_FIN\_LIST}    & 1.5l  130{0.5lg}         l \\
	\insignal{START\_LIST}      & l         0.5lg        66l \\
	\outsignal{BLOCK[127:0]}    & 66.5l         g          l \\
	\outsignal{FIN\_LIST}       & 66.5l         g          l \\
	\outsignal{CORRECT}         & 66.5l         g          l \\
	\extracode
	\tablerules
	\end{tikztimingtable}
\caption{\textit{block\_deserializer} -- odbieranie całego bloku}
\end{figure}


