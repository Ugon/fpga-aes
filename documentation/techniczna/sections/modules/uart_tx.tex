\newpage
\subsubsection{Moduł wysyłający bajty UART}
Moduł \textit{uart\_tx} wysyła bajty do klienta przez interfejs UART.

\begin{figure}[!h]
\begin{lstlisting}[style=vhdl, captionpos=b, caption={\textit{uart\_tx} -- interfejs modułu}]
port (
	reset_n               : in  std_logic;
	clk_16                : in  std_logic;
	tx                    : out std_logic;

	byte                  : in  std_logic_vector(byte_bits - 1 downto 0);
	start_transmitting    : in  std_logic;
	finished_transmitting : out std_logic);
\end{lstlisting}
\end{figure}

Opis sygnałów interfejsu modułu \textit{uart\_tx}:
\begin{interface}{FINISHED\_TRANSMITTING}
\item[\insignal{CLK\_16}] główny zegar układu.
\item[\insignal{START\_TRANSMITTING}] sygnał rozpoczynający wysyłanie danych.
\item[\insignal{BYTE[7:0]}] bajt do wysłania. Stabilność sygnału jest wymagana, gdy \insignal{START\_TRANSMITTING} jest w stanie wysokim.
\item[\outsignal{TX}] sygnał UART TX wychodzący do konwertera USB-UART.
\item[\outsignal{FINISHED\_TRANSMITTING}] sygnał sygnalizujący zakończenie wysyłania bajtu oraz gotowość na otrzymanie kolejnego sygnału \insignal{START\_TRANSMITTING} podczas kolejnego zbocza rosnącego.
\end{interface}

Moduł rozpoczyna transmisję po otrzymaniu sygnału \insignal{START\_TRANSMITTING} (rys. \ref{fig:uart-tx-stop-start}). Wysyłany jest bajt (rys. \ref{fig:uart-tx-frame}) dostarczony do modułu przez sygnał \insignal{BYTE[7:0]}. Bity startu, danych oraz stopu są wysyłane przez 16 cykli zegara \insignal{CLK\_16}. Zakończenie wysyłania ramki UART sygnalizowane jest przez sygnał \outsignal{FINISHED\_TRANSMITTING}, który również oznacza gotowość modułu na rozpoczęcie transmisji kolejnego bitu podczas następnego zbocza rosnącego. Sygnał \outsignal{FINISHED\_TRANSMITTING} wysyłany jest w przedostatnim cyklu zegarowym wysyłania bitu stopu, aby w ostatnim cyklu mógł nadejść następny sygnał \insignal{START\_TRANSMITTING}. Takie rozwiązanie umożliwia prowadzenie transmisji przez moduł z maksymalną możliwą prędkością -- ani jeden cykl zegara nie jest zmarnowany (rys. \ref{fig:uart-tx-stop-start}).

\newpage

\begin{center}
\centering
\resizebox{\textwidth}{!}{
	\begin{tikztimingtable}[timing/wscale=0.9]
	\insignal{CLK\_16}          & 3{cc}       16{cc}     16{cc}     3{cc}\\
	\insignal{START\_TRANS}     & 3U          15UH       16U        3U     \\
	\insignal{BYTE[7:0]}        & 3U          15UD       16U        3U          \\
	\outsignal{TX}              & 3D{Data[7]} 17.777K{Stop}  17.777J{Start} 3D{Data[0]}\\ %to offset crapy scaling
	\outsignal{FINISHED\_TRANS} & 3U          14UHU      16U        3U\\
	\extracode
	\tablerules
	\end{tikztimingtable}
}
\captionof{figure}{\textit{uart\_tx} -- wysłanie bitu stopu i startu}
\label{fig:uart-tx-stop-start}
\end{center}


\begin{center}
\centering
\resizebox{\textwidth}{!}{
	\begin{tikztimingtable}[timing/wscale=3.0]
	\insignal{CLK\_16}          & c              cc        cc         cc         cc         cc         cc         cc         cc         cc         cc       c \\
	\insignal{START\_TRANS}     & 0.5u0.5h 10.5U     \\
	\insignal{BYTE[7:0]}        & 0.5u0.5d 10.5U      \\
	\outsignal{TX}              & u              J{Start}  D{Data[0]} D{Data[1]} D{Data[2]} D{Data[3]} D{Data[4]} D{Data[5]} D{Data[6]} D{Data[7]} K{Stop}  u \\
	\outsignal{FINISHED\_TRANS} & u              9.5U 0.5h 1.5u\\
	\extracode
	\tablerules
	\end{tikztimingtable}
}
\captionof{figure}{\textit{uart\_tx} -- wysłanie całej ramki UART}
\label{fig:uart-tx-frame}
\end{center}