\subsubsection{Moduł serializujący bloki AES do bajtów UART}
Interfejs modułu serializującego bloki AES do bajtów UART \textit{block\_serializer}:

\begin{figure}[!h]
\begin{lstlisting}[style=vhdl]
port (
	reset_n                  : in  std_logic;
	clk_16                   : in  std_logic;

	tx_byte                  : out std_logic_vector(byte_bits - 1 downto 0);
	tx_start_transmitting    : out std_logic;
	tx_finished_transmitting : in  std_logic;

	aes_block                : in  std_logic_vector(block_bits - 1 downto 0);
	start_transmitting       : in  std_logic;
	finished_transmitting    : out std_logic);
\end{lstlisting}
\end{figure}

\begin{interface}{RX\_FINISHED\_TRANSMITTING}
	\item[\insignal{CLK\_16}] główny zegar układu.

	\item[\insignal{TX\_BYTE[7:0]}] sygnał wychodzący do komponentu \textit{uart\_tx}.
	\item[\outsignal{TX\_START\_TRANSMITTING}] sygnał pochodzący z komponentu \textit{uart\_tx}.
	\item[\insignal{TX\_FINISHED\_TRANSMITTING}] sygnał wychodzący do komponentu \textit{uart\_tx}.

	\item[\insignal{START\_TRANSMITTING}] sygnał rozpoczynający transmisję bloku.
	\item[\insignal{AES\_BLOCK[127:0]}] blok AES do wysłania. Stabilność sygnału jest wymagana, gdy \outsignal{START\_TRANSMITTING} jest w stanie wysokim.
	\item[\outsignal{FINISHED\_TRANSMITTING}] sygnał informujący o zakończeniu wysyłania bloku AES oraz gotowość na otrzymanie kolejnego sygnału \insignal{START\_TRANSMITTING} podczas kolejnego zbocza rosnącego zegara \insignal{CLK\_16}.
\end{interface}

Po otrzymaniu sygnału \insignal{START\_TRANSMITTING} moduł rozpoczyna transmisję. Wysyłanie bajtów bloku danych wykonywane jest przez przez moduł \textit{uart\_tx}. Kolejność wysyłanych bajtów jest zgodna z kolejnością ich odbierania przez moduł \textit{block\_deserializer}. W trakcie wysyłania obliczana jest suma kontrolna \textit{CRC16}. Po 128 bajtach bloku danych wysyłane są 2 bajty zawierające obliczoną sumę kontrolną. Po wysłaniu 130 bajtów moduł sygnalizuje gotowość na rozpoczęcie transmisji kolejnego bloku \outsignal{FINISHED\_LISTENING}.

\begin{figure}[!h]
	\centering
	\begin{tikztimingtable}[timing/wscale=0.95]
	\insignal{CLK\_16}           & 3{cc}            16{cc}           16{cc}         \\
	\outsignal{TX\_START\_TRANS} & LH               15LH             15LH           L\\
	\outsignal{TX\_BYTE[7:0]}    & LD{BLOCK[0]}     15LD{BLOCK[1]}   15LD{BLOCK[2]} L\\
	\insignal{TX\_FIN\_TRANS}    & L                15LH             15LH           2L\\
	\insignal{AES\_BLOCK[127:0]} & LD{BLOCK[127:0]} 16L              16L            L\\
	\insignal{START\_TRANS}      & LH               16L              16L            L\\
	\extracode
	\tablerules
	\end{tikztimingtable}
\caption{\textit{block\_serializer} -- rozpoczęcie wysyłania}
\end{figure}

\begin{figure}[!h]
	\centering
	\begin{tikztimingtable}[timing/wscale=0.95]
	\insignal{CLK\_16}           & 3{cc}           16{cc}       16{cc}  \\
	\outsignal{TX\_START\_TRANS} & 2LH             15LH         16L     \\
	\outsignal{TX\_BYTE[7:0]}    & 2LD{CRC[0]}     15LD{CRC[1]} 16L     \\
	\insignal{TX\_FIN\_TRANS}    & LH              15LH         15LH    L\\
	\outsignal{FIN\_TRANS}       & 2L              16L          15LH    L\\
	\extracode
	\tablerules
	\end{tikztimingtable}
\caption{\textit{block\_serializer} -- zakończenie wysyłania}
\end{figure}

\begin{figure}[!h]
	\centering
	\begin{tikztimingtable}
	\helpsignal{CLK\_UART}       & 1.5l  130{0.25c0.25c}    l \\
	\outsignal{TX\_START\_TRANS} & l     130{0.5lg}      1.5l \\
	\outsignal{TX\_BYTE[7:0]}    & l     130{0.5lg}      1.5l \\
	\insignal{TX\_FIN\_TRANS}    & 1.5l  130{0.5lg}         l \\
	\insignal{START\_TRANS}      & l         0.5lg        66l \\
	\insignal{AES\_BLOCK[127:0]} & l         0.5lg        66l \\
	\outsignal{FIN\_TRANS}       & 66.5l         g          l \\
	\extracode
	\tablerules
	\end{tikztimingtable}
\caption{\textit{block\_serializer} -- transmisja całego bloku}
\end{figure}






