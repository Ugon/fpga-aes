\newpage
\subsection{Skrypt klienta}
Skrypt klienta jest implementacją algorytmu komunikacji z serwerem szyfrującym (płytką Terasic DE1-SOC) przedstawionego w sekcji \ref{sec:przebieg-komunikacji}. Program przyjmuje od użytkownika tryb działania (szyfrowanie lub deszyfrowanie), klucz szyfrowania, wektor inicjalizacji oraz nazwy plików wejściowego i wyjściowego. Program odczytuje z dysku dane wejściowe, wysyła je do serwera oraz odbiera dane przetworzone i zapisuje je do pliku wynikowego. Jest napisany w języku python, oraz korzysta z bibliotek:
\begin{description}[noitemsep]
\item[serial] -- komunikacja przez port szeregowy UART
\item[threading] -- wielowątkowość związana z jednoczesnym wysyłaniem i odbieraniem bloków
\item[crcmod] -- obliczanie sumy kontrolnej crc16
\item[struct] -- konwersja liczby crc16 do tablicy bajtów możliwej do przesłania biblioteką serial
\item[argparse] -- parsowanie argumentów programu
\item[re] -- obsługa wyrażeń regularnych
\end{description}

\subsubsection{Padding}
Algorytm AES działa na blokach o wielkości 128b. W celu przetwarzania danych o wielkości nie będącej wielokrotnością rozmiaru bloku stosowany jest padding. W tym projekcie należy to do zadań klienta. Używany jest padding metodą dopełniania bloków do rozmiarów 128b bajtami o wartości równej brakującej liczbie bajtów. W celu uniknięcia niejednoznaczności przy usuwaniu wypełnienia, do wiadomości o wielkości będącej wielokrotnością 128b dodawany jest cały blok bajtów o wartości 16. W ten sposób ostatni bajt wiadomości zawsze zawiera liczbę bajtów dopełnienia, co pozwala na jednoznaczne usunięcie paddingu. 

Zastosowana metoda jest jedną ze standardowych praktyk. Została wybrana ze względu na fakt, że jest używana przez program \textit{openssl}, z którym produkt końcowy ma być kompatybilny. Jest zaimplementowana w sposób zgodny ze standardem RFC-2315 \cite{padding-rfc}.
