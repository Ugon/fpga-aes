\section{Dziedzina problemu}
\label{sec:dziedzina-problemu}

\subsection{Algorytm AES}

Advanced Encryption Standard (AES) jest symetrycznym szyfrem blokowym przyjętym przez NIST w 2001 roku jakos standard FIPS-197. Jest oparty na algorytmie Rijandael'a oraz występuje w trzech wariantach o różnych dłgościach kluczy (128, 192 lub 256 bitów). Bloki szyfru AES maja wielkość 128b, a do szyfrowania i deszyfrowania danych używany jest ten sam klucz.
\break
Algorytm AES operuje na blokach danych będących macierzami o wielkości 4x4 o elemntach będących bajtami bloku. Taką macierz będziemy również nazywać stanem.

\begin{center}
$\begin{bmatrix}
b_0 & b_4 & b_8    & b_{12} \\
b_1 & b_5 & b_9    & b_{13} \\
b_2 & b_6 & b_{10} & b_{14} \\
b_3 & b_7 & b_{11} & b_{15} \\
\end{bmatrix}$
\end{center}

Szyfrowanie AES składa się z iteracji, które są nazywane rundami. Liczba rund jest zależna od długości klucza.

\begin{center}
\begin{tabular}{|C{3cm}|C{3cm}|}
\hline
Długość klucza & Liczba rund\\
\hline
128 & 10\\
\hline
192 & 12\\
\hline
256 & 14\\
\hline
\end{tabular}
\end{center}

Rundy składają się z czterech podstawowych przekształceń operujących na stanie.
\begin{description}
\item[AddRoundKey] -- każdy bajt stanu jest zmodyfikowany przy pomocy operacji XOR z odpowiadającym mu bajtem klucza rundy.
\item[SubBytes] -- każdy bajt stanu jest zastąpiony odpowiadającym mu bajtem z \textit{Rijandael's S-box}.
\item[ShiftRows] -- trzy ostatnie rzędy stanu są cyklicznie przesunięte w lewo o odpowiednio 1, 2 lub 3 pozycje.
\item[MixColumns] -- każda z kolumn stanu zostaje pomnożona przez odpowiedni wielomian.
\end{description}

\subsubsection{Kroki algorytmu szyfrowania}
\begin{description}
\item[1. Rozwinięcie klucza] -- Każda z rund wymaga osobnego klucza o długości 128b. Oblicza się je na podstawie klucza wejściowego przy pomocy schematu Rijandael'a (ang. \textit{Rijandael key schedule}).
\item[2. Pierwsza runda] składająca się z jednego przekształcenia -- AddRoundKey
\item[3. Kolejne rundy] -- Każda z rund składa się z czterech przekształceń:
\begin{enumerate}[noitemsep,nolistsep]
\item SubBytes
\item ShiftRows
\item MixColumns
\item AddRoundKey
\end{enumerate}
\item[4. Ostatnia runda] składająca się z trzech przekształceń:
\begin{enumerate}[noitemsep,nolistsep]
\item SubBytes
\item ShiftRows
\item AddRoundKey
\end{enumerate}

\end{description}

\subsubsection{Deszyfrowanie}



\newpage