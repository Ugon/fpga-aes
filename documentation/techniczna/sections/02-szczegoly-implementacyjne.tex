\section{Szczegóły implementacyjne}
\label{sec:szczegoly-implementacyjne}

\subsection{Konfiguracja projektu oraz płytki z układem FPGA}
Układ FPGA na płytce Terasic DE1-SOC posiada zintegrowany procesor ARM (HPS, ang. \textit{Hard Processor System}). Dzięki temu możliwe jest m.in. uruchomienie systemu operacyjnego Linux. Między programowalną częścią układu a procesorem ARM jest interfejs umożliwiający szybką komunikacją. Pozwala to m.in. na skonfigurowanie układu FPGA jako karty graficznej i korzystania z systemu operacyjnego w trybie graficznym. Istotnymi dla tego projektu właściwościami takiego połączenia są:
\begin{enumerate}
\item Nie wszystkie urządzenia peryferyjne są podłączone bezpośrednio do programowalnej części układu FPGA -- niektóre są podłączone do procesora ARM. Powoduje to, że aby uzyskać do nich dostęp z FPGA wymagana jest dodatkowa konfiguracja multipleksacji pinów przeprowadzona przez preloader w jednej z faz startowych. Konwerter USB-UART firmy FTDI jest przykładem układu peryferyjnego, który jest podłączony do procesowa ARM i wymaga takie konfiguracji.
\item Obecność procesora ARM umożliwia wykonanie przy starcie płytki skryptów zdefiniowanych przez użytkownika i umieszczonych na karcie SD. Przykładem zastosowania jest programowanie układu FPGA przy starcie.
\end{enumerate}


\subsubsection{Fazy startowe}
Proces startowania płytki składa się z czterech faz \cite[p. 1068]{altera-vol3}\cite{rocketboards-preloader-uboot}. Dwie pierwsze fazy są konieczne do prawidłowego zainicjowania układu FPGA oraz zintegrowanego procesora ARM. Dwie kolejne fazy są opcjonalne. Kod wykonywalny fazy BootROM jest zintegrowany w HPS. Pozostałe fazy muszą być dostarczone przez użytkownika, np. na karcie SD.
\begin{enumerate}[noitemsep]
\item BootROM -- przeprowadza minimalną konfigurację oraz ładuje preloader do zintegrowanej pamięci RAM.
\item Preloader -- inicjalizacja SDRAM, konfiguracja multipleksacji pinów HPS I/O, załadowanie boot loadera do pamięci SDRAM.
\item Boot Loader (U-Boot) -- wykonuje zdefiniowane przez użytkownika skrypty startowe, ładuje system operacyjny.
\item Operating System
\end{enumerate}

\subsubsection{Konfiguracja pinów UART}
Do kowersji sygnału USB do UART postanowiłem wykorzystać znajdujący się na płytce Terasic DE1-SOC układ firmy FTDI. Jego sygnały nie są jednak podłączone bezpośrednio do programowalnej części układu FPGA, lecz do zintegrowanego procesora ARM. Aby uzyskać dostęp do jego sygnałów z FPGA należy odpowiednio skonfigurować projekt oraz stworzyć i wgrać na kartę SD preloader, który podczas uruchamiania płytki przestawi piny UART w odpowiedni tryb działania oraz określi ich kierunek.

\subsubsection{Skrypt startowy programujący układ FPGA}
Aby automatycznie zaprogramować układ FPGA podczas startu urządzenia stworzyłem skrypt, który będzie się wykonywał w fazie U-Boot.

Do programowania układu FPGA przy pomocy skryptu potrzebny jest plik konfiguracyjny w formacie \textit{.rbf}. Można go stworzyć przy pomocy środowiska Quartus na podstawie plików powstałych w wyniku kompilacji projektu.

\subsubsection{Przygotowanie karty SD}
Najłatwiejszym sposobem na stworzenie karty SD zawierającej potrzebne pliki jest modyfikacja dostarczonych na płycie CD do płytki Terasic DE1-SOC przykładowych obrazów. Lista kroków realizujących to zadanie przy użyciu komputera z systemem operacyjnum Linux:
\begin{enumerate}
\item Po włożeniu karty SD do czytnika należy określić jej ścieżkę w systemie operacyjnym, najlepiej analizując plik \textit{/proc/partitions}. Załóżmy, że ścieżką karty jest \textit{/dev/sdx}.
\begin{lstlisting}
  $ cat /proc/partitions
\end{lstlisting}

\item Wgrać dostarczony przez producenta obraz na kartę SD\cite{rocketboards-booting-prebuild}.
\begin{lstlisting}
  $ cat /proc/partitions
  $ sudo sync
\end{lstlisting}

\item Zastąpić domyślny preloader
\begin{lstlisting}
  $dupa
\end{lstlisting}

\item Wgrać skrypt U-Boot oraz plik zawierający konfigurację układu FPGA
\begin{lstlisting}
  $dupa
\end{lstlisting}



\end{enumerate}


\newpage