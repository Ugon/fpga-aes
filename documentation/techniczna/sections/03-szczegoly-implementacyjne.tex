\section{Szczegóły implementacyjne}
\label{sec:szczegoly-implementacyjne}



\subsubsection{Przygotowanie projektu oraz konfiguracja pinów UART}
Najwygodniejszym sposobem na rozpoczęcie projektu jest przystosowanie dołączonego przez producenta płytki projektu \textit{de1\_soc\_GHRD} (GHRD -- ang. \textit{Golden Hardware Reference Design}). GHRD jest projektem przeznaczonym do tego celu oraz jest skonfigurowany i gotowy do pracy z płytką Terasic DE1-SOC.
\break
Do kowersji sygnału USB do UART postanowiłem wykorzystać znajdujący się na płytce Terasic DE1-SOC układ firmy FTDI. Jego sygnały nie są jednak podłączone bezpośrednio do programowalnej części układu FPGA, lecz do zintegrowanego procesora ARM. Aby uzyskać dostęp do jego sygnałów z FPGA należy odpowiednio dostosować projekt oraz stworzyć i wgrać na kartę SD preloader, który podczas uruchamiania płytki przestawi piny UART w odpowiedni tryb działania oraz określi ich kierunek. Aby przystosować projekt należy \cite{altera-youtube-loanerio, altera-forum-fgga-hps-access, altera-forum-cant-rx}:
\begin{enumerate}
\item Otworzyć w edytorze QSys plik \textit{soc-system.qsys}
\item Wybrać komponent \textit{hps\_0} oraz w zakładce \textit{Peripheral Pins} dokonać zmian:
	\begin{itemize}[noitemsep,nolistsep]
	\item Zmienić wartość \textit{UART0 pin} na \textit{FPGA}
	\item Zmienić wartość \textit{UART0 mode} na \textit{Full}
	\item Zaznaczyć \textit{LOANIO 49} oraz \textit{LOANIO 50}
	\end{itemize}
\item Zapisać zmiany, oraz wygenerować kod VHDL klikając przycisk \textit{Generate HDL...}.
\item W głównym pliku projektu (ang. \textit{Top-Level Entity}) zmodyfikować komponent \textit{soc-system} tak, aby był zgodny z jego na nowo wygenerowaną wersją.
\item Przypisać wartość \textit{'1'} sygnałom:
	\begin{itemize}[noitemsep,nolistsep]
	\item \textit{hps\_0\_uart0\_cts}
	\item \textit{hps\_0\_uart0\_dsr}
	\item \textit{hps\_0\_uart0\_dcd}
	\item \textit{hps\_0\_uart0\_ri}
	\item \textit{hps\_0\_uart0\_rxd}
	\end{itemize}
\item Sygnał \textit{hps\_0\_hps\_io\_hps\_io\_gpio\_inst\_LOANIO49} przypisać do pinu \textit{HPS\_UART\_RX}.
\item Sygnał \textit{hps\_0\_hps\_io\_hps\_io\_gpio\_inst\_LOANIO50} przypisać do pinu \textit{HPS\_UART\_TX}.
\item Wektor sygnałów \textit{hps\_0\_h2f\_loan\_io\_oe} określa kierunek pinów pinów \textit{LOANIO}.
	\begin{itemize}[noitemsep,nolistsep]
	\item Do sygnału \textit{hps\_0\_h2f\_loan\_io\_oe(49)}, który odpowiada sygnałowi UART RX przypisać wartość \textit{'1'}, co odpowiada kierunkowi \textit{out}.
	\item Do sygnału \textit{hps\_0\_h2f\_loan\_io\_oe(50)}, który odpowiada sygnałowi UART TX przypisać wartość \textit{'0'}, co odpowiada kierunkowi \textit{in}.
	\end{itemize}
\end{enumerate}
Powyższe zmiany spowodują, że:
\begin{itemize}[noitemsep]
\item Sygnał \textit{hps\_0\_h2f\_loan\_io\_out(49)} będzie odpowiadał sygnałowi UART RX oraz będzie poprawnie połączony z konwerterem USB-UART firmy FTDI.
\item Sygnał \textit{hps\_0\_h2f\_loan\_io\_out(50)} będzie odpowiadał sygnałowi UART TX oraz będzie poprawnie połączony z konwerterem USB-UART firmy FTDI
\end{itemize}

\subsubsection{Przygotowanie preloadera}
Aby przy starcie płytki piny UART zostały poprawnie połączone z układem FPGA należy z plików powstałych w wyniku kompilacji projektu wygenerować preloader. Aby to zrobić posłużyłem się narzędziem \textit{bsp-editor} będącym częścią oprogramowania Quartus, zgodnie z instrukcją znalezioną w internecie \cite{rocketboards-preloader}.

\subsubsection{Skrypt startowy programujący układ FPGA}
Aby automatycznie zaprogramować układ FPGA podczas startu urządzenia stworzyłem skrypt, który będzie się wykonywał w fazie U-Boot \cite{rocketboards-uboot-script}.
\begin{lstlisting}
fatload mmc 0:1 $fpgadata soc_system.rbf;
fpga load 0 $fpgadata $filesize;
\end{lstlisting}
Do programowania układu FPGA przy pomocy skryptu potrzebny jest plik konfiguracyjny w formacie \textit{.rbf}. Można go stworzyć konwertując przy pomocy środowiska Quartus plik \textit{.sof} powstały w wyniku kompilacji projektu \cite{rocketboards-sof-to-rfb}. Użyłem trybu programowania \textit{Fast Passive Parallel X16}, któremu odpowiada ustawienie przełączników MSEL[0:4] znajdujących się na płytce na wartość 00100.

\subsubsection{Przygotowanie karty SD}
Najłatwiejszym sposobem na stworzenie karty SD zawierającej potrzebne pliki jest modyfikacja dostarczonych na płycie CD do płytki Terasic DE1-SOC przykładowych obrazów. Lista kroków realizujących to zadanie przy użyciu komputera z systemem operacyjnym Linux:
\begin{enumerate}
\item Po włożeniu karty SD do czytnika należy określić jej ścieżkę w systemie operacyjnym, najlepiej analizując plik \textit{/proc/partitions}. Załóżmy, że ścieżką karty jest \textit{/dev/sdx}.
\begin{lstlisting}
  $ cat /proc/partitions
\end{lstlisting}

\item Wgrać dostarczony przez producenta obraz \textit{DE1\_SoC\_SD.img} na kartę SD \cite{rocketboards-booting-prebuild}.
\begin{lstlisting}
  $ sudo dd if=DE1\_SoC\_SD.img of=/dev/sdx bs=1M
  $ sudo sync
\end{lstlisting}

\item Zastąpić domyślny preloader
\begin{lstlisting}
  $ sudo dd if=preloader-mkpimage.bin of=/dev/sde3 bs=64k seek=0
  $ sudo sync
\end{lstlisting}

\item Wgrać skrypt U-Boot oraz plik zawierający konfigurację układu FPGA na partycję FAT \textit{/dev/sdx1} dowolnym sposobem (np. korzystając z systemowego eksploratora plików).
\end{enumerate}

Aby przy starcie płytki programowanie FPGA przebiegło pomyślnie, należy umieścić kartę SD w czytniku oraz ustawić przełączniki MSEL[4:0] w pozycji 00100.

\subsubsection{Główny zegar}
Główny zegar \textit{CLK\_16}, którym taktowany będzie projektowany układ ma częstotliwość
\begin{equation}
f_{CLK\_16} = 16 * UART\_BAUD\_RATE
\end{equation}
gdzie \textit{UART\_BAUD\_RATE} jest szybkością transmisji sygnału UART wyrażoną w baudach, która dla tego projektu wynosi 115200 baudów. Taka wartość podyktowana jest faktem, że częstą praktyką interpretacji sygnału UART, którą postanowiłem zastosować, jest jego próbkowanie z częstotliwością 16 razy szybszą niż częstotliwość zmian sygnału.
\break
Częstotliwość zegara \textit{CLK\_16} jest o wiele niższa niż dostępny w układzie FPGA zegar o częstotliwości 50MHz. Pozwoliło to na uzyskanie żądanej częstotliwości przy pomocy prostego dzielnika częstotliwości.
\break
Układ jest wyzwalany rosnącymi zboczami zegara \textit{CLK\_16}. Jeśli akcja mająca miejsce w danym rosnącym zboczu zależy od wartości innego sygnału, to ma on stałą wartość co najmniej od poprzedzającego do następnego zbocza malejącego.


\subsubsection{Synchronizacja sygnału UART RX}
Sygnał UART RX pochodzi z peryferyjnego konwertera USB-UART firmy FTDI, który jest w domenie innego zegara. Powoduje to, że w momencie wystąpienia zbocz rosnących lub malejących głównego zegara, sygnał RX może nie mieć dobrze określonej wartości (np. również być w trakcie zbocza). Może to prowadzić do wystąpienia stanów metastabilnych \cite{altera-metastability} i w efekcie doprowadzić do niedeterministycznego działania układu. Aby temu zapobiec zsynchronizowałem sygnał RX, prowadząc go przez dwa przerzutniki typu D \cite{altera-metastability, 2ff-synchronization} wyzwalane głównym zegarem. Jest to standardowa technika, która powoduje drastyczne zmniejszenie prawdopodobieństwa wystąpienia stanów metastabilnych z powodu używania sygnałów pochodzących z domen innych zegarów.

\subsubsection{Moduł odbierający bajty UART}
Interfejs układu odbierającego bajty UART \textit{uart\_rx}:

\begin{interface}{FINISHED\_LISTENING}
	\item[\insignal{CLK\_16}] główny zegar układu.
	\item[\insignal{RX}] zsynchronizowany przy pomocy dwóch przerzutników typu D sygnał UART RX, wychodzący z konwertera USB-UART.
	\item[\insignal{START\_LISTENING}] sygnał wprowadzający komponent w stan oczekiwania na bajt.
	\item[\outsignal{BYTE[7:0]}] odebrany bajt. Stabilność sygnału jest gwarantowana, gdy \outsignal{FINISHED\_LISTENING} jest w stanie wysokim.
	\item[\outsignal{FINISHED\_LISTENING}] sygnał informujący o zakończeniu odbierania bajtu oraz gotowość na otrzymanie kolejnego sygnału \insignal{START\_LISTENING} podczas kolejnego zbocza rosnącego zegara \insignal{CLK\_16}.
\end{interface}

Po otrzymaniu sygnału \insignal{START\_LISTENING} moduł rozpoczyna nasłuchiwanie na transmisję. Każdy bit jest próbkowany 16 razy. Wartość bitu jest ustalana na podstawie \textit{głosowania} -- bit jest uznawany za {'1'} jeśli co najmniej dwie z trzech środkowych próbek mają wartość {'1'}, analogicznie dla {'0'}. Transmisja bajtu zostaje uznana za rozpoczętą, jeśli zostanie odebrany bit startu -- gdy zostanie napotkana pierwsza próbka o wartości {'0'} i potwierdzona przez głosowanie. Bajt zostaje uznany za poprawnie odebrany, jeśli zostanie odebrany poprawny bit stopu. Bit stopu jest próbkowany jedynie 9 razy. Jest to minimalna liczba próbek pozwalająca przeprowadzić \textit{głosowanie}. Próbkowanie bitu stopu jest skrócone, ze względu na fakt, że pomimo ustawienia zgodnych parametrów transmisji nadajnika i odbiornika, zegar urządzenia nadającego może być minimalnie szybszy. Prowadzi to do zbyt wolnego odbierania napływających informacji i po kilku odebranych bajtach prowadzi do błędów, co zostało zaobserwowane w trakcie testów. Skrócenie czasu odbierania bitu stopu zapobiega takim sytuacjom.

\begin{figure}[!h]
	\centering
	\begin{tikztimingtable}
	\insignal{CLK\_16}          & 32{cc}  \\
	\insignal{RX}               & HHH    16J{Start}    13D{Data[0]}\\
	\insignal{START\_LISTENING} & LH30L\\
	\extracode
	\tablerules
	\draw[red, ->] (3.5,0) -- (3.5,1);
	\draw[red, ->] (9.5,0) -- (9.5,1);
	\draw[red, ->] (10.5,0) -- (10.5,1);
	\draw[red, ->] (11.5,0) -- (11.5,1);
	\draw[red, ->] (26.5,0) -- (26.5,1);
	\draw[red, ->] (27.5,0) -- (27.5,1);
	\draw[red, ->] (28.5,0) -- (28.5,1);
	\end{tikztimingtable}
\caption{\textit{uart\_rx} -- odbiór bitu startu}
\end{figure}

\begin{figure}[!h]
	\centering
	\begin{tikztimingtable}
	\insignal{CLK\_16}              & 31{cc}  \\
	\insignal{RX}                   & 13D{Data[7]} 16J{Stop} HH   \\
	\outsignal{BYTE[7:0]}           & 22U D 8U \\
	\outsignal{FINISHED\_LISTENING} & 22L H 8L \\
	\extracode
	\tablerules
	\draw[red, ->] (4.5,0) -- (4.5,1);
	\draw[red, ->] (5.5,0) -- (5.5,1);
	\draw[red, ->] (3.5,0) -- (3.5,1);
	\draw[red, ->] (19.5,0) -- (19.5,1);
	\draw[red, ->] (20.5,0) -- (20.5,1);
	\draw[red, ->] (21.5,0) -- (21.5,1);
	\end{tikztimingtable}
\caption{\textit{uart\_rx} -- odbiór bitu stopu}
\end{figure}

\begin{figure}[!h]
	\centering
	\begin{tikztimingtable}[timing/wscale=2.8]
  	\insignal{CLK\_UART}            & c cc        cc         cc         cc         cc         cc         cc         cc         cc         cc       c \\
  	\insignal{RX}                   & u J{Start}  D{Data[0]} D{Data[1]} D{Data[2]} D{Data[3]} D{Data[4]} D{Data[5]} D{Data[6]} D{Data[7]} K{Stop}  u \\
  	\outsignal{BYTE[7:0]}           & 10U 0.5d 1.5u \\
	\outsignal{FINISHED\_LISTENING} & 10L 0.5h 1.5l \\
	\extracode
	\tablerules
	\end{tikztimingtable}
\caption{\textit{uart\_rx} -- odbiór całej ramki UART}
\end{figure}
\newpage
\subsection{Moduł wysyłający bajty UART}
Moduł \textit{uart\_tx} wysyła bajty do klienta przez interfejs UART.

\begin{figure}[!h]
\begin{lstlisting}[style=vhdl, captionpos=b, caption={\textit{uart\_tx} -- interfejs modułu}]
port (
	reset_n               : in  std_logic;
	clk_16                : in  std_logic;
	tx                    : out std_logic;

	byte                  : in  std_logic_vector(byte_bits - 1 downto 0);
	start_transmitting    : in  std_logic;
	finished_transmitting : out std_logic);
\end{lstlisting}
\end{figure}

Opis sygnałów interfejsu modułu \textit{uart\_tx}:
\begin{interface}{FINISHED\_TRANSMITTING}
\item[\insignal{CLK\_16}] główny zegar układu.
\item[\insignal{START\_TRANSMITTING}] sygnał rozpoczynający wysyłanie danych.
\item[\insignal{BYTE[7:0]}] bajt do wysłania. Stabilność sygnału jest wymagana, gdy \insignal{START\_TRANSMITTING} jest w stanie wysokim.
\item[\outsignal{TX}] sygnał UART TX wychodzący do konwertera USB-UART.
\item[\outsignal{FINISHED\_TRANSMITTING}] sygnał sygnalizujący zakończenie wysyłania bajtu oraz gotowość na otrzymanie kolejnego sygnału \insignal{START\_TRANSMITTING} podczas kolejnego zbocza rosnącego.
\end{interface}

Moduł rozpoczyna transmisję po otrzymaniu sygnału \insignal{START\_TRANSMITTING} (rys. \ref{fig:uart-tx-stop-start}). Wysyłany jest bajt (rys. \ref{fig:uart-tx-frame}) dostarczony do modułu przez sygnał \insignal{BYTE[7:0]}. Bity startu, danych oraz stopu są wysyłane przez 16 cykli zegara \insignal{CLK\_16}. Zakończenie wysyłania ramki UART sygnalizowane jest przez sygnał \outsignal{FINISHED\_TRANSMITTING}, który również oznacza gotowość modułu na rozpoczęcie transmisji kolejnego bitu podczas następnego zbocza rosnącego. Sygnał \outsignal{FINISHED\_TRANSMITTING} wysyłany jest w przedostatnim cyklu zegarowym wysyłania bitu stopu, aby w ostatnim cyklu mógł nadejść następny sygnał \insignal{START\_TRANSMITTING}. Takie rozwiązanie umożliwia prowadzenie transmisji przez moduł z maksymalną możliwą prędkością -- ani jeden cykl zegara nie jest zmarnowany (rys. \ref{fig:uart-tx-stop-start}).

\newpage

\begin{center}
\centering
\resizebox{\textwidth}{!}{
	\begin{tikztimingtable}[timing/wscale=0.9]
	\insignal{CLK\_16}          & 3{cc}       16{cc}     16{cc}     3{cc}\\
	\insignal{START\_TRANS}     & 3U          15UH       16U        3U     \\
	\insignal{BYTE[7:0]}        & 3U          15UD       16U        3U          \\
	\outsignal{TX}              & 3D{Data[7]} 17.777K{Stop}  17.777J{Start} 3D{Data[0]}\\ %to offset crapy scaling
	\outsignal{FINISHED\_TRANS} & 3U          14UHU      16U        3U\\
	\extracode
	\tablerules
	\end{tikztimingtable}
}
\captionof{figure}{\textit{uart\_tx} -- wysłanie bitu stopu i startu}
\label{fig:uart-tx-stop-start}
\end{center}


\begin{center}
\centering
\resizebox{\textwidth}{!}{
	\begin{tikztimingtable}[timing/wscale=3.0]
	\insignal{CLK\_16}          & c              cc        cc         cc         cc         cc         cc         cc         cc         cc         cc       c \\
	\insignal{START\_TRANS}     & 0.5u0.5h 10.5U     \\
	\insignal{BYTE[7:0]}        & 0.5u0.5d 10.5U      \\
	\outsignal{TX}              & u              J{Start}  D{Data[0]} D{Data[1]} D{Data[2]} D{Data[3]} D{Data[4]} D{Data[5]} D{Data[6]} D{Data[7]} K{Stop}  u \\
	\outsignal{FINISHED\_TRANS} & u              9.5U 0.5h 1.5u\\
	\extracode
	\tablerules
	\end{tikztimingtable}
}
\captionof{figure}{\textit{uart\_tx} -- wysłanie całej ramki UART}
\label{fig:uart-tx-frame}
\end{center}
\subsubsection{Moduł deserializujący bajty UART do bloków AES}
Interfejs modułu deserializującego bajty UART do bloków AES \textit{block\_deserializer}:
\begin{interface}{RX\_FINISHED\_LISTENING}
	\item[\insignal{CLK\_16}] główny zegar układu.

	\item[\insignal{RX\_BYTE[7:0]}] sygnał pochodzący z komponentu \textit{uart\_rx}.
	\item[\outsignal{RX\_START\_LISTENING}] sygnał wychodzący do komponentu \textit{uart\_rx}.
	\item[\insignal{RX\_FINISHED\_LISTENING}] sygnał pochodzący z komponentu \textit{uart\_rx}.

	\item[\insignal{START\_LISTENING}] sygnał wprowadzający komponent w stan oczekiwania na blok.
	\item[\outsignal{BLOCK[127:0]}] odebrany blok AES. Stabilność sygnału jest gwarantowana, gdy \outsignal{FINISHED\_LISTENING} jest w stanie wysokim.
	\item[\outsignal{FINISHED\_LISTENING}] sygnał informujący o zakończeniu odbierania bloku AES oraz gotowość na otrzymanie kolejnego sygnału \insignal{START\_LISTENING} podczas następnego zbocza rosnącego zegara \insignal{CLK\_16}.
	\item[\outsignal{CORRECT}] sygnał określający, czy transmisja bloku AES przebiegła bez błędów. Stabilność sygnału jest gwarantowana, gdy \outsignal{FINISHED\_LISTENING} jest w stanie wysokim.
\end{interface}

Po otrzymaniu sygnału \insignal{START\_LISTENING} moduł rozpoczyna nasłuchiwanie na transmisję. Odbieranie bajtów bloku danych wykonywane jest przez przez moduł \textit{uart\_rx}. Kolejność ułożenia bajtów w bloku \outsignal{BLOCK[127:0]} jest zgodna ze standardem AES. W trakcie odbierania obliczana jest suma kontrolna \textit{CRC16}. Po 128 bajtach bloku danych przesyłane są 2 bajty zawierające oczekiwaną, obliczoną po stronie klienta sumę kontrolną. Jeśli jest ona zgodna z tą obliczaną na bieżąco przez moduł, sygnał \outsignal{CORRECT} przyjmuje wartość {'1'}, w przeciwnym wypadku wartość {'0'}. Po odebraniu 130 bajtów moduł zwraca blok \outsignal{BLOCK[127:0]} wraz z informacją o jego poprawności \outsignal{CORRECT} oraz sygnalizuje gotowość na rozpoczęcie nasłuchiwania na kolejne bloki \outsignal{FINISHED\_LISTENING}.

\begin{figure}[!h]
	\centering
	\begin{tikztimingtable}[timing/wscale=0.95]
	\insignal{CLK\_16}          & 3{cc}  16{cc}           16{cc}         \\
	\outsignal{RX\_START\_LIST} & LH     15LH             15LH           L\\
	\insignal{RX\_BYTE[7:0]}    & L      15LD{BLOCK[0]}   15LD{BLOCK[1]} 2L\\
	\insignal{RX\_FIN\_LIST}    & L      15LH             15LH           2L\\
	\insignal{START\_LIST}      & LH     16L              16L            L\\
	\extracode
	\tablerules
	\end{tikztimingtable}
\caption{\textit{block\_deserializer} -- rozpoczęcie odbierania}
\end{figure}

\begin{figure}[!h]
	\centering
	\begin{tikztimingtable}[timing/wscale=0.95]
	\insignal{CLK\_16}          & 3{cc}          16{cc}       16{cc}             \\
	\outsignal{RX\_START\_LIST} & 2LH            15LH         16L                \\
	\insignal{RX\_BYTE[7:0]}    & LD{BLOCK[127]} 15LD{CRC[0]} 15LD{CRC[1]}       L\\
	\insignal{RX\_FIN\_LIST}    & LH             15LH         15LH               L\\
	\outsignal{BLOCK[127:0]}    & 3L             15L          15LD{BLOCK[127:0]} L\\
	\outsignal{FIN\_LIST}       & 3L             15L          15LH               L\\
	\outsignal{CORRECT}         & 3L             15L          15LH               L\\
	\extracode
	\tablerules
	\end{tikztimingtable}
\caption{\textit{block\_deserializer} -- zakończenie odbierania}
\end{figure}

\begin{figure}[!h]
	\centering
	\begin{tikztimingtable}
	\helpsignal{CLK\_UART}      & 1.5l  130{0.25c0.25c}    l \\
	\outsignal{RX\_START\_LIST} & l     130{0.5lg}      1.5l \\
	\insignal{RX\_BYTE[7:0]}    & 1.5l  130{0.5lg}         l \\
	\insignal{RX\_FIN\_LIST}    & 1.5l  130{0.5lg}         l \\
	\insignal{START\_LIST}      & l         0.5lg        66l \\
	\outsignal{BLOCK[127:0]}    & 66.5l         g          l \\
	\outsignal{FIN\_LIST}       & 66.5l         g          l \\
	\outsignal{CORRECT}         & 66.5l         g          l \\
	\extracode
	\tablerules
	\end{tikztimingtable}
\caption{\textit{block\_deserializer} -- odbieranie całego bloku}
\end{figure}







\newpage