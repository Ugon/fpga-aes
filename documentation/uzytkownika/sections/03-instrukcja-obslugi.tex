\section{Instrukcja obsługi systemu}
Program komputerowy można uruchomić korzystając z terminala, a jego obsługa jest analogiczna do programu \textit{openssl}.

Składnia polecenia:
\begin{verbatim}
sudo ./fpga-aes [-d] -in input_file -out output_file -K key -iv IV
\end{verbatim}

Program trzeba uruchomić z uprawnieniami użytkownika root, ponieważ korzysta z portu szeregowego. Program przyjmuje parametry:
\begin{interface}{-out}
\item[\textbf{-d}]Flaga oznaczająca działanie w trybie deszyfrowania. Jeśli nie jest podana to program działa w trybie szyfrowania.
\item[\textbf{-in}]Nazwa pliku wejściowego
\item[\textbf{-out}]Nazwa pliku wyjściowego
\item[\textbf{-K}]Klucz szyfrowania
\item[\textbf{-iv}]Wektor inicjalizacji
\end{interface}