\section{Instrukcja obsługi systemu}
Program komputerowy można uruchomić korzystając z terminala, a jego obsługa jest analogiczna do programu \textit{openssl}.

Składnia polecenia:
\begin{verbatim}
sudo ./fpga-aes [-d] -in input_file -out output_file -K key -iv IV
\end{verbatim}

Program trzeba uruchomić z uprawnieniami użytkownika root, ponieważ korzysta z portu szeregowego. Program przyjmuje parametry:
\begin{interface}{-out}
\item[\textbf{-d}]Flaga oznaczająca działanie w trybie deszyfrowania. Jeśli nie jest podana to program działa w trybie szyfrowania.
\item[\textbf{-in}]Nazwa pliku wejściowego
\item[\textbf{-out}]Nazwa pliku wyjściowego
\item[\textbf{-K}]Klucz szyfrowania zapisany jako liczba w systemie szesnastkowym (256 bitów, 64 cyfry szesnastkowe)
\item[\textbf{-iv}]Wektor inicjalizacji zapisany jako liczba w systemie szesnastkowym (128 bitów, 32 cyfry szesnastkowe)
\end{interface}

\subsection{Przykłady uruchomienia}
W celu zaszyfrowania pliku o nazwie \textit{plaintext-file.txt} do pliku \textit{cyphertext} kluczem \textit{000102030405060708090a0b0c0d0e0f101112131415161718191a1b1c1d1e1f} oraz z wektorem inicjalizacji \textit{12345678901234567890123456789012} należy uruchomić program poleceniem:

\begin{verbatim}
sudo ./fpga-aes -in plaintext-file.txt -out cyphertext 
-K 000102030405060708090a0b0c0d0e0f101112131415161718191a1b1c1d1e1f 
-iv 12345678901234567890123456789012
\end{verbatim}

Aby ten plik odszyfrować, należy użyć polecenia:

\begin{verbatim}
sudo ./fpga-aes -d -in cyphertext -out plaintext-file.txt 
-K 000102030405060708090a0b0c0d0e0f101112131415161718191a1b1c1d1e1f 
-iv 12345678901234567890123456789012
\end{verbatim}